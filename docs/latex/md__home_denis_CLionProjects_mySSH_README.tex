\subsection*{Introduction}

my\+S\+SH is a minimalistic cryptographic network protocol for remote control of a system over an unsecured network. A my\+S\+SH server will listen on a machine waiting for a number of clients to connect over a netwrok and securely trasmit commands. The commands (also called command lines) will have a specific format and the server will ensure the correct execution and forwading of the output to the client. Installation

\subsection*{Documentation}

Please read my\+S\+S\+H.\+pdf in the docs directory

\subsection*{Dependencies}


\begin{DoxyEnumerate}
\item Install Open\+S\+SL on your distribution
\end{DoxyEnumerate}

\subsection*{Installation}


\begin{DoxyEnumerate}
\item In the main directory, run {\ttfamily make -\/B server}
\item In the main directory, run {\ttfamily make -\/B client}
\item In the main directory, run {\ttfamily make -\/B shell}
\end{DoxyEnumerate}

\subsection*{Usage}

\subsubsection*{\hyperlink{classServer}{Server}}


\begin{DoxyEnumerate}
\item Make bin folder the current directory and run server with the port to run.
\item Wait for clients to connect and be served Example\+: {\ttfamily sudo ./server 2018}
\end{DoxyEnumerate}

\subsubsection*{Client}


\begin{DoxyEnumerate}
\item Run the client using ip, port and username. Example\+: {\ttfamily ./client 127.\+0.\+0.\+1 2018 denis} 
\end{DoxyEnumerate}